%!TEX root = thesis.tex
In the previous chapters we have investigated our initial interpretation of AHIs though inspirational user studies and prototyping.
Our initial thoughts on AHIs were based on existing research directions and inspiration from the domestic domain which led us the idea of AHIs as a novel interaction approach that, in our optics, touches upon some little unexplored areas of HCI.  
As we have now gone through three very different approaches to creating AHIs it is now time to zoom out from the individual details of the approaches and revisit our concept of AHIs based on our experiences, both the successful and the less successful.

\section{Affordances}
\section{Feedback and Feedforward}

\section{Knowledge construction}
\todo{eller research perspective eller \dots}

\begin{verbatim}
1. Projektet i lyset af Zimmermans 4 linser
* Process, invention, relevans, extensibility

2. perspektiver til ``Strong concepts''
* generativitet
* abstraktion

maybe a 4.th construction approach, ``augmenting existing objects ad hoc'', pinoky, REVEL, textile touch

a look back: the interfaces have embedded function or meaning outside of interaction
\end{verbatim}

\section{Fourth approach}
In our exploratory evaluation of Textile Touch we saw a pointer to a way of approaching AHIs that we had not considered in our initial outlay of the concept.

\todo{reference back to evaluation, someone points to the fact of enhancing existing objects} 

Where we with the Textile Touch prototype has focused on it as an example of how interaction can be integrated into the existing environment in a ubiquitous fashion, it can also be used as an example of how to extend the capabilities of existing objects and augment its functions in an ad hoc manner.

At a glance this does sound somewhat like Augmented Reality (AR), but as \citet{azuma1997survey} describes it AR \emph{allows the user to see the real world, with virtual objects superimposed upon or composited with the real world.}
So the focus is on augmenting the environment virtually, while our focus is on augmenting the environment physically.

\todo{more on why this is so}

Inspired by this idea of ``augmenting existing objects in an ad hoc fashion'' we have taken a brief look around for other examples that might give weight to this approach.


REVEL \citep{bau2013revel}, as we have briefly mentioned earlier, is a system that via a digital overlay augments objects by changing the tactile perception of their physical surface.
So it is a physical augmentation that, in principle, can be done to arbitrary objects giving them a new function when part of the interaction, while retaining the existing properties of the object \todo{reformulate - focus on the aspect of the object being useful while augmented}

Another example is Pinoky \citep{sugiura2012pinoky} that is a wireless ring-like device that can augment ordinary plush toys to allow them to move limbs.
This is done by attaching the ring around a limb and the Pinoky can now record movements done by the user and play them back, somewhat like earlier described Topobo \citep{raffle2004topobo}.

