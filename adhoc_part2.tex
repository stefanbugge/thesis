%!TEX root = thesis.tex
In the previous chapters we have investigated our initial interpretation of AHIs though inspirational user studies and prototyping.
Our initial thoughts on AHIs were based on existing research directions and inspiration from the domestic domain which led us the idea of AHIs as a novel interaction approach that, in our optics, touches upon some little unexplored areas of HCI.  
As we have now gone through three very different approaches to creating AHIs it is now time to zoom out from the individual details of the approaches and revisit our concept of AHIs based on our experiences, both the successful and the less successful.

\section{Affordances}
\section{Feedback and Feedforward}

\section{Knowledge construction}
\todo{eller research perspective eller \dots}

\begin{verbatim}
1. Projektet i lyset af Zimmermans 4 linser
* Process, invention, relevans, extensibility

2. perspektiver til ``Strong concepts''
* generativitet
* abstraktion


\end{verbatim}
