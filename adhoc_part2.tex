%!TEX root = thesis.tex
In the previous chapters we have investigated our initial interpretation of AHIs though inspirational user studies and prototyping.
Our initial thoughts on AHIs were based on existing research directions and inspiration from the domestic domain which led us the idea of AHIs as a novel interaction approach that, in our optics, touches upon some little unexplored areas of HCI.  
As we have now gone through three very different approaches to creating AHIs it is now time to zoom out from the individual details of the approaches and revisit our concept of AHIs based on our experiences, both the successful and the less successful.

\section{Affordances}
\begin{quotation}
\emph{When Koffka asserted that ``each thing says what it is'' he failed to mention that it may lie. More exactly, a thing may not look like what it is \citep{gibson1979ecological}.}
\end{quotation}
The above quote from \citeauthor{gibson1979ecological} is especially interesting in relation to AHIs as it points to a challenge that we have not really touched upon in this thesis.
In the following we will take a look at the concept of affordances to assert whether it makes sense to talk about affordances in dynamic interfaces such as SCIs and AHIs.
\subsubsection{Affordances}
The notion of affordances describes the relationship between the acting organism and the acted-upon environment.
The term comes from the above cited perceptual psychologist J.J. Gibson.
Affordances are properties of the environment that, for good or ill, tells us about the possible actions that are available to us.
Affordances therefore describe the potentials for actions in the environment as we perceive that environment.  

\subsubsection{Perception and affordances}
Perception is a key concept in relation to affordances as in order to ``know'' an affordance is there, you have to perceive it with your senses. Where \citet{gibson1979ecological} focuses mostly on visual perception, \citet{gaver1991technology} wants to broaden it to include all the senses, such as tactile and auditory feedback.
For example, sound can be used to indicate affordances that cannot be seen, where the sound of a turning lock indicates that the affordance of the door has changed.

\citet{gaver1991technology} notes that \emph{``Affordances per se are independent of perception''}.
This separation of affordances and the information available about them to us, gives rise to complications, as affordances will still exists even if you do not perceive them or if you perceive them wrong.
Also you might perceive an affordance that does not exists.
\blank
\todo{Possible perspectives}
\begin{verbatim}
Dynamic affordances - dynamic form allows for dynamic affordances (SCI, AHI)

Norman: 
 Mapping (additive and substitutive), 
 When the number of possible functions exceed the 
  number of controls, there apt to be difficulty
 ''each control is just where it ought to be''
 Feedback 

Djajadiningrat:
 Semantic vs Direct affordances (textile touch - jamming)

Warren:
 construction of affordances (conductive paint) 
Gaver:
 sequential and nested affordances (alle tre)
\end{verbatim}

\todo{to wensveen - interaction frogger, feedback and feedforward}

\section{Knowledge construction}
\todo{eller research perspective eller \dots}

\begin{verbatim}
1. Projektet i lyset af Zimmermans 4 linser
* Process, invention, relevans, extensibility

2. perspektiver til ``Strong concepts''
* generativitet
* abstraktion

maybe a 4.th construction approach, ``augmenting existing objects ad hoc'', pinoky, REVEL, textile touch

a look back: the interfaces have embedded function or meaning outside of interaction

Fishkin - taxonomy of tangible interfaces
\end{verbatim}

\section{Perspectives}
\todo{maaske anden overskrift}

\citet{abowd2012next} notes that, while many of Weiser's predications about the future of computing has come true, some were not so accurate such as inch-scaled devices being readily disposable.
While this is true, our concept of AHIs does, in some aspects, challenge this as we propose that AHIs \emph{disappears} after use.
We see this disappearance both in the sense of the affordance or perception of the interface, so not a physical disposal in the sense that \citeauthor{abowd2012next} is referring to, but we also see the disappearance as a physical removal of the interface.
This will most likely be most apparent in our third construction approach to AHIs as we there suggest that interfaces are build on the spot which will, at least in some cases, also lead to deconstruction of the interface after use and therefore be disposable in the sense \citeauthor{abowd2012next} is referring to.

As we saw with our exploration of paint as a construction mechanism the interface could be disposed simply with a wet cloth.
From a sustainability point of view disposable interfaces are of course not a \hl{positive} approach so such interfaces should instead focus on reuse, in our paint interface this could for example mean that the paint is collected in a container for later use instead of simply being washed away. 

\section{A possible fourth approach}
In our exploratory evaluation of Textile Touch we saw a pointer to a way of approaching AHIs that we had not considered in our initial outlay of the concept.

\todo{reference back to evaluation} 

Where we with the Textile Touch prototype have focused on it as an example of how interaction can be integrated into the existing environment in a ubiquitous fashion, it can also be used as an example of how to extend the capabilities of existing objects and augment its functions in an ad hoc manner.

At a glance this does sound somewhat like Augmented Reality (AR), but as \citet{azuma1997survey} describes it AR \emph{allows the user to see the real world, with virtual objects superimposed upon or composited with the real world.}
So the focus is on augmenting the environment virtually, while our focus is on augmenting the environment physically.

\todo{more on why this is so}

Inspired by this idea of ``augmenting existing objects in an ad hoc fashion'' we have taken a brief look around for other examples that might give weight to this approach.

REVEL \citep{bau2013revel}, as we have briefly mentioned earlier, is a system that via a digital overlay augments objects by changing the tactile perception of their physical surface.
So it is a physical augmentation that, in principle, can be done to arbitrary objects giving them a new function when part of the interaction, while retaining the existing properties of the object \todo{reformulate - focus on the aspect of the object being useful while augmented}

Another example is Pinoky \citep{sugiura2012pinoky} that is a wireless ring-like device that can augment ordinary plush toys to allow them to move limbs.
This is done by attaching the ring around a limb and the Pinoky can now record movements done by the user and play them back, somewhat like earlier described Topobo \citep{raffle2004topobo}.

