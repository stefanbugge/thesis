%!TEX root = ../thesis.tex

\textbf{Chapter 1} is an introductory chapter where we give an overview of the domain of this thesis as well as a motivation for our work and the ideas that have inspired us.
\blank
\textbf{Chapter 2} gives an overview of the evolution of computers and their interfaces.
This overview enables us to situate ourself in a broader context of computer science and point to an area that we see possibilities in.  
\blank
\textbf{Chapter 3} is a discussion about the domestic environment as a domain and is used to provide a context for our design explorations. 
\blank
\textbf{Chapter 4} presents our initial notion of Ad Hoc Interfaces (AHIs) as a novel approach to interface design.
Here we present and situate AHIs in relation to existing literature and tendencies that were presented in chapter \ref{ch:ui} and \ref{ch:domain}, and in relation to existing designs that show ad hoc qualities.
We also envision three different approaches to constructing AHIs that we will explore in chapter \ref{ch:jamming}-\ref{ch:proto3}.
\blank
\textbf{Chapter 5} explores the design space of AHIs in the domestic environment through workshops conducted in the home.
\blank
\textbf{Chapter 6} explores the first approach to constructing AHIs. 
We describe our first prototype work that explores and surveys the possibility of creating AHIs based on shape-change and jamming techniques and present various concepts based on this.
\blank
\textbf{Chapter 7} describes and discusses the secondary construction approach where we, though several iterations of a prototype, explore the possibilities of creating embedded AHIs through electronic textiles.
\blank
\textbf{Chapter 8} explores the third approach where interfaces are constructed on the spot.
We discuss the approach based on concept designs and exploratory prototypes. 
Of the three approaches this is the most cursory.
\blank
\textbf{Chapter 9} zooms out from the specific design and concept explorations and returns to our notion of Ad Hoc Interfaces.
Our initial presentation of AHIs was mostly theory-driven, but we now complement it with a more \emph{research through design} oriented approach based on chapters \ref{ch:workshops}-\ref{ch:proto3}. 
\blank
\textbf{Chapter 10} concludes our thesis by both looking backwards to what we have achieved and contributed with, but also with a look forward, pointing to future work and directions that AHIs could benefit from.
\blank
We have chosen not to include a separate related work chapter in this thesis and related work will therefore be presented and related to throughout the thesis.
It is our hope that this will enable us to make the related work more relevant and context specific and therefore support our points better. 