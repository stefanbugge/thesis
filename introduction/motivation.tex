%!TEX root = ../thesis.tex
In this chapter we will attempt to bring together some of the ideas that has laid the ground and motivation for our thesis work.
Our process has not been on an entirely straight line, so at the same time we will try to draw lines between the different areas and disciplines that have shaped and influenced our process and ideas into the end result we present in this thesis.  

The original basis for this thesis was born from a fascination of things that \emph{moves, adapts and changes} - the bringing of ``life'' to the objects and environments that surrounds us, letting them transform to our needs, both in function and form. 

The fascination of giving life to inanimate objects is not entirely new.
A famous example of this is the Vaucanson Duck, an automata created by the French inventor and artist Jacques Vaucanson in 1739 \citep{riskin2003defecating}.
It was later depicted by a nineteenth-century inventor, as seen in figure~\ref{vaucanson_duck} 
The mechanical duck, which supposedly contained over a thousand parts, could both flap its wings and appeared to have the ability to eat, digest, and defecate grains.

The animation of `things' still amazes today, both in the world of technology, with advancements in robotics, and in the world of crafts where ingenuity and craftsmanship can give rise to fascination.   
An example of the latter is Theo Jansen's that gives life to new species of animals with his amazing Strandbeests \cite{strandbeestJansen}.  
Made mostly from yellow plastic tube and fabric sails, these skeletons traverses the beaches of the Netherlands, living off the wind, adapting to the turnings of the elements, see figure~\ref{strandbeest}.

\begin{figure}[h]
	\centering
	\begin{minipage}{.45\textwidth}
		\centering
		\includegraphics[width=0.9\linewidth]{figures/strandbeest}
		 \captionof{figure}{One of Jansen's Strandbeests, \citep{strandbeestJansen}}
		\label{strandbeest}
	\end{minipage}%
	\hspace{0.1cm}
	\begin{minipage}{.45\textwidth}
		\centering
		\includegraphics[width=0.9\linewidth]{figures/vaucanson_duck}
		\captionof{figure}{A nineteenth-century inventor's imagined depiction of the inner workings of the Vaucanson Duck \citep{riskin2003defecating}}
		\label{vaucanson_duck}
	\end{minipage}	
\end{figure}
These two examples, each impressive in their own right, points to the powerful expressiveness that actuation can give objects, as we have a tendency to relate things that move to living entities. 
\blank
In traditional computing systems there has been a strong separation between the digital bits and the material world surrounding us.
The tangibility of digital information has for the most part been limited to keyboards and mice, icons and windows. 
And even though metaphors and graphical representations has helped us visualize the digital matter that flow through the transistors, these digital bits, when presented to the physical world, still only manifests as pixels on a screen, intangible and transient.
At the same time this represents one of the forces of digital information and digital interfaces, as their intangible and transient nature give them a degree of malleability and adaptability that are not found in the physical non-living objects that we surrounds us with. 
\blank
In 1965, Ivan Sutherland presented a vision of what he called the Ultimate Display \citep{sutherland1965ultimate}, envisioning a room where digital bits would control the existence of matter, completely removing the boundaries between the two worlds.
This display would attain the palpability of the physical world as well as the transience of the digital, as it would let digital information manifest itself into physical objects.
This is, of course, still only a vision, but in recent years there has been an increasing interest in bridging the gap between the two worlds.
Under different names such as kinetic interaction, organic user interfaces, actuated interfaces, shape-changing interfaces and programmable matter, researchers has attempted to get closer to creating truly ubiquitous systems.    
This has resulted in an increasing amount of of what \citet{coelho2009programming} calls `transient materials', such as flexible displays, shape-changing materials, e-textiles and sensor networks.
\blank
Interestingly some of these transient materials, especially e-textiles, have given rise to DIY communities that combines traditional crafts with electronics, \hl{appealing(andet ord)} to a new form of material end-user-programming.
DIY is interesting as a concept as it challenges the traditional balance between products and users making it possible for the consumer to create  or modify what he/she wants, instead of relying on the producer or designer define what is wanted.
This does somewhat democratize the product development, as is also seen in the digital world with open-source software.
\blank
The idea of giving the control back to the user does also relate to the relationship between computers and user.
In ubiquitous computing, especially the ones focused on context-awareness, there is a tendency to exclude the user from the control-loop as interaction possibilities and action execution are depended on the \emph{sensed} context, which not necessarily correlates with what the user finds as the right \emph{correct} context.
So this gives rise to the question: how do we give back the control to user?
\blank
hjemmet
\blank
\todo{afrunding herfra, udkast}

In the midst of all these seemingly divergent digressions we believe that there is an unexplored space for interfaces that focuses on adaptability and situational awareness, that function on the terms of the user and not the computer system, bringing a more dynamic relationship, as seen in software, into the world of physical interactable artefacts.  

Noter:
\begin{verbatim}
Hvad er status nu og her og hvad er mulighederne
- ``men, der er faktisk et rum som ikke er udforsket'' - det tidslige, situerede

Computerfokus vs brugerfokus
- Contextual interfaces vs Ad hoc interfaces
- Hvordan kan man give brugeren kontrollen tilbage (i modsaetning til CA)
- Refererende perspektiv - Participatory design - demokratisere - empower the user
- Eksempelvis: end-user-programming
- Tilgangen peger paa et design rum

Hjemmet
- Hvad er et hjem
- Vi fokuserer paa hjemme - en valgt begraensning/omraade med potentiale
- S. Brand og struktur
- Hvad hjemmet betyder for dem der bor der

DIY og infrastruktur
- Balance mellem producent og bruger
- Traditionelle crafts (fra Coelho)

Digital vs fysisk
- Materialitet (shapechange, digital bits)


Afslutte med en aabning til projektet - hvordan vi med udgangspunkt i ovenstaaende har bevaeget os videre, hvad vil vi udforske.
\end{verbatim}