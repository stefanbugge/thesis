%!TEX root = thesis.tex
The third approach to creating AHIs that we presented in chapter~\ref{ch:adhoc} was to construct the interface on the spot.
Of the three approaches this is the one that we have explored the least.

----This approach originated from the idea of interfaces that was ``fashioned from whatever is immediately available''.
--While this may not be quite possible, the idea is engaging and intriguing as an embodiment of a very literal version of AHIs. 
we presented this approach as interfaces that was .. .. .. 
But how do you go about making such a system?

There exist various physical ``frameworks'' that incorporates at least some of the aspects that we see in this approach to creating AHIs.
\todo{examples}

We have done a few, though a bit cursory, experiments to explore the concept.
The main challenge has been to find materials, or a generic enough platform, that allows you to create objects or environments with interactive capabilities from \emph{whatever is immediately available}. 
\blank
We have explored the use of Bare Conductive, an electric water-based paint, as a material for creating ad hoc interfaces on the spot.
This electrical paint has some interesting properties as it is easily applied and dries quickly but at the same time, because it is water-based, can be removed with a wet cloth.
This fits well with our idea of quick to make, quick to erase.

We see two overall options for using a paint like this.
Firstly as a connector where the paints either acts as a wire between two points or as a switch - both able to be painted on.
Secondly as a variable resistor

\begin{verbatim}
physical programming - digital programming
related work : topobo, surflex, music blocks
bare paint exploration
the need for infrastructure - will always be limited by the underlying system
leading to AHIs
Inspired by the domain, bring the infrastructure out from the walls,
 aesthetics and functionality
 deconstruct
\end{verbatim}
