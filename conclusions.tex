%!TEX root = thesis.tex
\todo{Conclusions from each chapter}
In this thesis we have argued for Ad Hoc Interfaces as a novel approach to user-computer interaction and as a addition to the existing field of HCI. \todo{lidt bedre indledning}

Chapters 1-3 was used as a seedbed for our notion of AHIs, where we motivated, presented and discussed different approaches to user interfaces, human-computer relations and a specific application domain.
We pointed to areas where we saw potential for furthering the vision of ubiquitous computing, with a focus on dynamic interfaces that connect the physical world and the digital world.
We also argued that our contribution could be a potential fit for a possible 4th generation of computing.  

In chapter 4 we presented our initial interpretation(andet ord?) of AHIs.
We deduced a definition based on exiting literature (chapter 1-3) and based on existing interfaces that, in our opinion, exhibited ad hoc characteristics.
We also presented an earlier project of ours, BeoMotion, in which we identified many of the qualities lead us to the notion of AHIs.
We defined AHIs as 

\begin{quotation}
\emph{Tangible interfaces that can be created or accessed on demand for a particular purpose and with the user as the initiator of action. When in use the interface will temporarily emerge to the foreground after which it disappears into the background, either physically or perceptually.}
\end{quotation}

This puts emphasises a dynamic relation between user and interface


\section{Future work}

\todo{Some open research question}

\begin{verbatim}
Research questions - 5 nye specialer

use research through design \citet{zimmerman2007research} as ofset to conclusion,
  process, invention, relevance, extensibility 

\end{verbatim}


