%!TEX root = thesis.tex

In this thesis we have argued for Ad Hoc Interfaces as a novel approach to user-computer interaction and as an addition to the existing field of HCI.

Chapters 1--3 were used as a seedbed for our notion of AHIs, where we motivated, presented and discussed different approaches to user interfaces, human-computer relations and a specific application domain.
We pointed to areas where we saw potential for furthering the vision of ubiquitous computing, with a focus on dynamic interfaces that connect the physical world and the digital world.
We also argued that our contribution could be a potential fit for a possible 4th generation of computing.  

In chapter 4 we presented our initial definition of AHIs.
This definition was based on our review of existing literature (chapter 1--3) and on highlighting existing interfaces which, in our opinion, exhibit ad hoc characteristics.
We also presented an earlier project of ours, BeoMotion, in which we identified many of the qualities that lead us to the notion of AHIs.
We defined AHIs as 

\begin{quotation}
\emph{Tangible interfaces that can be created or accessed on demand for a particular purpose and with the user as the initiator of action. When in use the interface will temporarily emerge to the foreground after which it disappears into the background, either physically or perceptually.}
\end{quotation}
In extension to this definition we also proposed three concrete approaches to building AHIs in practice.

Before we embarked on the exploration of the three approaches we did preliminary workshops to get a deeper insight into the design space of the home and to explore the dynamic material and interaction approach that we suggested with AHIs.
This was followed up with a separate exposition of each of the three approaches.

While in the first half of the thesis our approach was dominated by existing practice and literature, chapters 6--8 took a more research through design approach where we, through conceptualising and prototyping, focused on practical experiences with AHIs.
In our design process we explored a diverse set of approaches such as jamming, touch sensitive textiles and conductive paint.
Our experiences from this process formed the basis for revisiting our original notion of AHIs as to extend and modify it based on our practical experiences with construction, conceptualisation and user evaluation.
We presented a modified definition and added a fourth possible approach to the creation of AHIs.
Here we also took a perspectival look at how interaction with AHIs can be understood and the challenge that dynamic objects pose for such an approach.
This was done in chapter 9.
\blank
In Zimmerman et al.'s \citep{zimmerman2007research} work on integrating design in research and practice, they state four criteria for evaluating research through design, \emph{process}, \emph{invention}, \emph{relevance} and \emph{extensibility}.

Throughout our \emph{process} we have experimented with and explored three distinct approaches to AHIs where each approach in detail has been documented both in terms of concepts and prototypes, and in terms of practical knowledge and experience as to the construction of such concepts and prototypes.
It is our hope that we, through our documentation of the technical challenges we have faced, can allow others to push the prospects of AHIs even further.

In terms of \emph{invention} we have argued for a novel approach to the designing of human-computer interfaces called Ad Hoc Interfaces.
Based on a review of existing HCI areas we have situated our approach in relation to ubiquitous computing, tangible interfaces, context-aware computing and shape-changing interfaces.
Through AHIs we have argued for a more dynamic relationship between users and interfaces, an approach that, in our opinion, touches upon important aspects of our future interaction with computers.

We have argued for the \emph{relevance} of our AHI approach both in terms of the general development of user interfaces, where we stand before the possible emergence of the 4th generation of computing, but also in terms of how users can integrate with computers in the domestic environment, an area where DIY also plays a role.

Lastly we have enabled \emph{extensibility} through a definition, and four different construction approaches, of AHIs and by documenting the technical and theoretical background for our prototypes and concepts for others to extend and experiment with.

\section{Future directions}
A key aspect of interfaces such as we have presented in this thesis is the reliance on dynamic or \emph{transient} material, which is still a research area very much in its infancy.
So any advances in transient material could potentially benefit AHIs and enable even more novel interaction possibilities than what we have suggested.
Still, we have only explored a fraction of possible approaches to dynamic interfaces (jamming, e-textiles and conductive paint), so even with today's technology there is still a plethora of possibilities for exploring AHIs, such as soft mechanics, nano technology, biological electronics, flexible displays and so on.

As we see it, future work in AHIs can take many different directions.
The suggested approaches to the construction of AHIs are not nearly exhausted, and could each in on their own be the basis for a new thesis.
The area of understanding the interaction with AHIs also calls for further work as we have only touched the surface here, pointing to the Interaction Frogger framework for an offset to approach dynamics interfaces such as AHIs.

In this thesis we have only touched upon a small part of what AHIs potentially could offer, in terms of new types of user experiences and new ways to integrate technology into the physical environment.
We see lots of potential in materials and interfaces that provide a more dynamic relationship between the digital and physical world, as we then can utilise more of the qualities of digital interfaces, such as transience, adaptability and customisation, and bring it into the physical world.

