%!TEX root = thesis.tex
\todo{Conclusions from each chapter}

In this thesis we have argued for Ad Hoc Interfaces as a novel approach to user-computer interaction and as a addition to the existing field of HCI. \todo{lidt bedre indledning}

Chapters 1-3 was used as a seedbed for our notion of AHIs, where we motivated, presented and discussed different approaches to user interfaces, human-computer relations and a specific application domain.
We pointed to areas where we saw potential for furthering the vision of ubiquitous computing, with a focus on dynamic interfaces that connect the physical world and the digital world.
We also argued that our contribution could be a potential fit for a possible 4th generation of computing.  

In chapter 4 we presented our initial interpretation(\todo{andet ord?}) of AHIs.
We deduced a definition based on exiting literature (chapter 1-3) and based on existing interfaces that, in our opinion, exhibited ad hoc characteristics.
We also presented an earlier project of ours, BeoMotion, in which we identified many of the qualities lead us to the notion of AHIs.
We defined AHIs as 

\begin{quotation}
\emph{Tangible interfaces that can be created or accessed on demand for a particular purpose and with the user as the initiator of action. When in use the interface will temporarily emerge to the foreground after which it disappears into the background, either physically or perceptually.}
\end{quotation}
In extension to this definition we also proposed three concrete approaches to building AHIs in practice.

Before we embarked on the exploration of the three approaches we did preliminary workshops to get a deeper insight into the design space of the home and to explore the dynamic material and interaction approach that we suggested with AHIs.
This was followed up with a separate exposition of each the three approaches.

Where we in the first half of the thesis had an approach that was dominated by existing practice and literature, chapters 6-8 took a more research through design approach where we, though conceptualising and prototyping, focused on practical experiences with AHIs.
In our design process we explored a diverse set of approaches such as jamming, touch sensitive textiles and conductive paint.
Our experiences from this process formed the basis for revisiting our original notion of AHIs as to extend and modified it based on our practical experiences with construction, conceptualisation and user evaluation.
Here we also took a perspectival look at affordances and the challenge that dynamic objects pose for such an approach.
This was done in chapter 9.
\blank
\citet{zimmerman2007research} states four criteria for evaluating research through design, process, invention, relevance and extensibility.
In this thesis we have experimented and explored three distinct approaches to AHIs where each approach in detail have been documented


\section{Future directions}
We have only touched upon a small part of what AHIs potentially could offer in this thesis, in terms of new types of user experiences and new way to integrate technology into the physical environment.
We see lots of potential in materials and interfaces that provides a more dynamic relationship between the digital and physical world, as we then can utilise more of the qualities of digital interfaces, such as transience, adaptability and customisation, and bring it into the physical world.


\todo{Some open research question}

\begin{verbatim}
Research questions - 5 nye specialer

use research through design \citet{zimmerman2007research} as ofset to conclusion,
  process, invention, relevance, extensibility 

\end{verbatim}


