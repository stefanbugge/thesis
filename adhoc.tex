%!TEX root = thesis.tex

In this thesis we will introduce the notion of \emph{ad hoc interfaces} or AHIs.

Being \emph{ad hoc} generelly means something made for a specific task or creating meaning in changing contexts.
The word itself originates from the Latin language and literally means \emph{for this} (situation). 
In modern english the two word combination is regarded as a single word and the Oxford Dictionary\footnote{http://oxforddictionaries.com/definition/english/ad-hoc} defines it as

\begin{quotation}
\textbf{Ad hoc}  /ad 'h\textturnscripta k/

created or done for a particular purpose as necessary
\end{quotation}

The word is used in many different fields and context.
For example, In society committees a formed on an ad hoc basis to deal with specific tasks, investigations or analyses.

In computer science wireless networking can be of an ad hoc nature, which means that that the network is not dependent on a preexisting infrastructure.
Wireless ad hoc networks are highly dynamic and all participating nodes in the network graph are more or less doing routing for each other.
This makes the network adaptive to changing contexts, which is an important element of ubiquitous computing.

In general a user works with a computer in an ad hoc manner.
General purpose computers are very good at supporting ad hoc activities with their multitasking capabilities.
Applications are used when needed and maybe even used in connection with each other.
Windows are opened, closed and arranged as needed to support different use cases.

So, ad hoc is a prevalent characteristic within many activities and contexts.
Next we will give our own definition of being ad hoc when applied to the field of user interfaces and also compare this to other elements of \todo{computer science}.

\section{Ad hoc interfaces} 
We define ad hoc interfaces as \todo{physical} interfaces that can be created or accessed on demand for a particular purpose in mind. 

In this sense there is a degree of temporality embedded in the definition, in that the creation is somewhat impromptu and the purpose is non-continuing.

The questions is, how do we apply ad hoc elements or characteristics to physical interfaces.
The physical world comes with many constraints compared to the digital.
We can't just magically make physical objects appear and disappear which is to do on a digital display.
\todo{But what can we do} 
\blank
Let's first have a look at some existing products and projects that exhibit ad hoc characteristics. 
\todo{\ensuremath{\leftarrow} overgang til naeste subsektion}

\subsection{Products with ad hoc characteristics}
Ad hoc characteristics in interfaces are not some novel invention and they an be found in several existing products.
A simple example would be The Clapper~(figure~\ref{ch:adhoc:theclapper})~\cite{theclapperWIKIPEDIA} from the mid eighties, an electrical switch reacting to sounds in a specific frequency, tuned to claps, to turn a switch on and off respectively.
Here the interface is pervasive in the nearby environment and one interacts with it when needed after which it ``disappears'' again.

\begin{figure}[hb]
	\centering
  		\includegraphics[width=1in]{figures/theclapper}
	\caption[The Clapper, a sound activated electronic switch.]
   {The Clapper, a sound activated electronic switch.}
   \label{ch:adhoc:theclapper}
\end{figure}

Another example is the conceptual and somewhat futuristic product ShapePhone (figure~\ref{fig:ch:jamming:jui-phone}) based on particle jamming, which will also be addressed in \autoref{ch:jamming}.
ShapePhone is generic shape changing product which changes it behaviour based on its physical form.
For example, in its base form it is a phone but when wrapped around a wrist it could serve as a watch and when folded in some other way it could be a game controller.
So, the different interfaces are created on demand and have very different purposes of use.

In a masters course in 2011 called Innovation Project \cite{innoproj2011, beomotionreportstefan, beomotionreporttore}  we, the authors, designed and implemented a dynamic shape shifting wall module with the ability to change its surface structure for acoustic regulation, i.e. diffusion, absorption and reflection for optimal conditions.
On top of that the module had illuminating features so that different areas could light up and serve as an aesthetic lighting for the atmosphere of the room. 
We envisioned two scenarios of use for the wall.
One where context awareness was the primary driver which meant that everything (acoustic and lighting conditions) would be automatically sensed and controlled.
The other where a user to a higher degree would be in control.
The user could actively create deformations on the surface for aesthetic purposes and also control illuminated areas by interacting with the wall surface, see \ref{fig:ch:adhoc:beomotion}. \todo{afrunding a paragraf} 
\blank
\todo{2-3 flere eksempler \dots} 

\begin{figure}
	\centering
	\begin{subfigure}{.46\textwidth}
		\centering
		\includegraphics[width=.9\linewidth]{figures/beomotion/prototype}
		\caption{Interactive prototype}
	\end{subfigure}%
	\begin{subfigure}{.54\textwidth}
		\centering
		\includegraphics[width=.9\linewidth]{figures/beomotion/concepts}
		\caption{Concept illustrations}
	\end{subfigure}
	\caption{BeoMotion product design, Innovation Project, Aarhus University 2011}
	\label{fig:ch:adhoc:beomotion}
\end{figure}

\subsubsection{Comparisons to other (X)UI} 
It might seem that AHI overlaps with the concept of Context Aware Computing in that they both have a focus on environmental context.
In context aware computing a system attempts to derive, through a variety of cues, what the current context of use is and as a result it adapts its behaviour \citep[chap. 8]{krumm2009ubiquitous}. 
In this way it is the system itself that takes action autonomously and the user continues on outside of the control loop. \todo{rephrase??}
Exactly the topic of control is one of the points where context aware systems have received criticism \cite{erickson2002some}, \citep[chap. 8]{krumm2009ubiquitous}.
The criticism has to do with a systems ability to make inference based on the analysis of quantitative contextual information available, something that can be quite difficult to do for a computer as contextual information is often subtle and implicit.
\todo{This is not supposed to be a critique of context aware systems as they surely have their legitimacy and offer a lot of convenience.}
\todo{An AHI can have CAC elements} 

Where context aware systems put the user out of the control loop AHIs do not.
In AHIs the user is the initiator of action.

It actually makes no sense to compare them directly as they cannot be juxtaposed.
They are both elements or characteristics of a system and one does not exclude the other as a part of a system.
It would seem quite natural to have an ad hoc system that also incorporates some context aware computing elements, for example \todo{give a good example} 
\blank
\todo{and TUIs}
\blank
\todo{and NUIs}  
\blank

\section{Discussion}

Some of the more interesting or ideal examples may seem 

\subsubsection{Noter \dots  } 
\begin{verbatim}

``fashioned from whatever is immediately available''

bringing together different aspects of various fields within HCI.

\end{verbatim} 
