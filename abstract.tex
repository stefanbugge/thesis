%!TEX root = thesis.tex

In this thesis we define our notion of \emph{ad hoc interfaces} (AHIs) as a novel approach to tangible user interfaces, which focuses on a more dynamic relationship between users and interfaces.
%baseret paa
An initial definition is derived and based on a literature review and by highlighting ad hoc characteristics from existing interfaces, including a preceding project we made.
AHIs are tangible interfaces that are created or accessed for a particular purpose with a focus on the user as the initiator of action.
This definition is used as the basis for a more design oriented exploration of AHIs which serves to uncover the potentials of such an approach to user interfaces.

Following the definition, three distinct construction approaches are formulated, pointing to concrete approaches for realising AHIs, guiding further explorations throughout the thesis.

%udforsket gennem
Three design implementations are presented, each addressing one of the formulated construction approaches to AHIs.
These approaches comprise using shape-change, embedding invisible interfaces into the environment, and the literal construction on the spot.
These approaches are explored through jamming, e-textiles, and conductive paint respectively. 
Each approach is addressed through a review of relevant related work and approaches and exploratory work through conceptualisation and prototypes.
%resultater
User studies are conducted to explore the design space of AHIs in the domestic environment.
User evaluations and the exploratory work shed light on both the technical challenges and potentials as well as concrete application areas.

%revisiting
Based on our exploratory work we revisit our original definition of AHIs to discuss perspectives and adjustments of the definition.
This includes a proposed fourth construction approach, where the focus is on augmenting existing interfaces or objects, as well as a reflection on how users perceive such interfaces.

\chapter*{Resum\'e}
\addcontentsline{toc}{chapter}{Resum\'e}

I dette speciale definerer vi vores beskrivelse af \emph{ad hoc interfaces} (AHI'er) som en ny tilgang til håndgribelige brugergrænseflader, der opfordrer til et mere dynamisk forhold mellem brugere og grænseflader.
Der præsenteres en definition der er baseret på en litteraturgennemgang samt ved en fremhævelse af ad hoc egenskaber i eksisterende grænseflader, herunder også et af vores forgående projekter.
AHI'er er håndgribelige grænseflader, der skabes eller tilgåes på baggrund af et specifikt formål og med et fokus på brugeren som initiativtager til handling.

I forbindelse med definitionen formuleres tre forskellige konstruktionstilgange, som peger på konkrete tiltag for realiseringer af AHI'er.
Disse tilgange tjener som retningslinjer for den videre udforskning i specialet.

Igennem specialet bliver tre design-implementationer præsenteret, som hver især adresserer en af de formulerede konstruktionstilgange til AHI'er.
Disse tilgange omfatter brugen af form-skift, indlejring af usynlige grænseflader i omgivelserne og en nu-og-her konstruktion på stedet.
Hver tilgang adresseres gennem en gennemgang af relevant, relateret arbejde, samt gennem eksplorativt arbejde med konseptualiseringer og prototyper.
Disse tilgange er udforsket henholdsvis gennem jamming, e-tekstiler, og strømledende maling.
Brugerundersøgelser er foretaget for at udforske designrummet for AHI'er i hjemmet.
Brugerevalueringer og det eksplorative arbejde belyser både tekniske udfordringer og potentialer samt konkrete applikationsområder.

Baseret på vores eksplorative arbejde vender vi tilbage til vores oprindelige definition af AHI'er og diskuterer perspektiver og tilpasninger af definitionen.
Dette inkluderer blandt andet en refleksion over, hvordan brugere opfatter sådanne grænseflader, samt et forslag til en fjerde konstruktionstilgang, hvor fokus er på at udvide eller tilpasse eksisterende grænseflader eller objekters funktioner.
