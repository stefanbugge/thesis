%!TEX root = thesis.tex
\todo{Overall introduction and possibly a new headline instead of Domain}
\section{The evolution of buildings}
Buildings, like everything else, changes over time.
But different parts of a building change with different rates, a wall will most likely endure longer than the paint on its sides and the ground on which the wall stands will most likely still be there when the wall collapses.
This ever changing nature of buildings has been conceptualized by Steward Brand in his book \emph{How Buildings Learn: What Happens After They're Built} \citep{brand1995buildings}.
Here Brand presents a framework, in which he define six S's as a ground for understanding changes in a building: Site, Structure, Skin, Services, Space plan and Stuff.

\begin{figure}[hb]
	\centering
  		\includegraphics[width=3in]{figures/brand-diagram}
	\caption[The different layers of change, taken from \cite{brand1995buildings}]
   {The different layers of change \cite{brand1995buildings}}
   \label{brand-diagram}
\end{figure}

The six S's describes the different layers of change, three describing the exterior: Site, Structure and Skin and three describing the interior: Services, Space plan and Stuff.
Each layer has a different rate of change, Site having the longest and Stuff the shortest rate of change, see figure~\ref{brand-diagram}. As the rate of change differs for the different components of the building, the building is, as Brand notes, constantly `tearing itself apart', or in a more constructive term, constantly evolving.

Brand argues in favor for an approach where the inhabitants of a building can evolve and change the building over time according to their needs \citep{brandBBCvideo}.
He sees this in contrast to a scenario where a single person or group designs a building for others to use.
In light of this, providing inhabitants with higher rates of change could accommodate an even more dynamic relationship between the building and the people living in it.
Inspired by this concept of the building as a living, ever-evolving entity, we want to explore how we can accommodate and design for change in physical interfaces in the context of digital systems in a domestic setting.

We want to explore how physical interfaces can be created on demand in an ad hoc manner, to better cope with the changes in the buildings space plan and stuff, and better accommodate the changing needs of its inhabitants.
The idea is to let the interface exist when it is needed and then let it slowly perish when it is no longer used, either manually, over time, by natural forces or based on sensory data.
This does somewhat challenge the traditional view of physical interfaces as they are generally considered static and permanent.
To put this into perspective we will give an overview of the development of interfaces in chapter~\ref{ch:ui}. \todo{hvis kap. 2 og 3 byttes om skal dette tilpasses}

\section{Smart homes or home that make us smart}
The idea of adding computer systems into the domestic environment is not a new one.
Ever since the emergence of electricity into the households, electrical appliances has been a part of the domestic setting.
And along with the increasingly more advanced appliances being developed, the dream and vision of even more sophisticated systems have continuously evolved

An early visual example of the fascination of `smart homes' is seen in the General Motors commercial film \emph{Design for Dreaming} from 1956 \citep{designfordreaming,designfordreamingWIKIPEDIA}, where the \emph{Kitchen of the Future} is presented \citep{homeofthefutureWIKIPEDIA}.
Here we see a remote control of the different kitchen appliances, a (computer)screen that can show the final result of the recipe it is given, and a seemingly context-aware moving table.
Domestic technology appears even earlier where in the 1915-20, with the advent of electricity in common homes, electrically powered machines such as vacuum cleaners and refrigerators providing the `seedbed', as Frances Aldrich puts it, for the emergence for the smart home \citep{aldrich2003smart}.

In recent years, especially since Mark Weiser's introduction of Ubiquitous Computing \citep{weiser1991computer}, a lot of the research in the domestic realm has focused on the idea of a \emph{home that is smart} where the goal is to use computers to make the home \emph{intelligent} \citep{taylor2007homes}.
\citet{aldrich2003smart} defines smart homes as:

\begin{quotation}
\emph{A ``smart home'' can be defined as a residence equipped with computing and information technology which anticipates and responds to the needs of the occupants, working to promote their comfort, convenience, security and entertainment through the management of technology within the home and connections to the world beyond.}
\end{quotation}
Aldrich sees this definition as the acme of domestic technology as we can envision it today putting it somewhere in between reality and fantasy.
\todo{mere her}

The domestic setting differs a lot from the traditional office setting where most research on ubiquitous computing has been conducted, and is in some ways even more complex as people in the domestic setting has a lot more freedom as to how they organize their space and time \cite{meyer2003survey}.
Also the aspect of user experience needs special consideration when introducing new technology to the home, whereas in a work setting the user might \emph{need} to use the new technology, imposed upon him by superiors, the technology for the home need to be usable and useful enough so that the user \emph{wants} to use \citep{meyer2003survey}.
A thorough look at the challenges related to creating ubiquitous computing for the home is presented by \citet{edwards2001home}.
Here Edwards et al. identifies seven challenges that need to be overcome before the vision of the ubiquitous smart home can become a reality. We will return to these challenges in relation to our own prototypes \todo{ref til prototype sektion}.

\citet{rodden2003evolution} have built upon Brands framework in the context of ubiquitous computing and the domestic setting and is therefore relevant for us here.
By looking at existing research Rodden et al. identifies three main approaches for creating interactive devices for domestic settings.

\begin{itemize}
  \item \emph{Information Appliance} that are stand-alone, self-contained devices that are often used as a layer of interactive functions on top of an existing appliance.
  \item \emph{Interactive Household Objects} that embeds interactive capabilities into existing household objects to create new means of interaction and communication, often building upon the existing understanding and metaphors associated with the household object.
  \item \emph{Augmented Furniture} that embeds sensory systems into furniture to add interactive capabilities. \ldots
\end{itemize}

Rodden et al. notes that the technology is most intrusive in information appliances, less so in interactive household objects and least in augmented furniture.

\todo{noget om bygninger ikke er nybyggerier og noget mere Rodden}

\todo{diskussion om empoverment, Brand/Rodden vs smart-homes/context-awareness/dubai-huse}

\section{Ad hoc interfaces}

In this thesis we will introduce the notion of \emph{ad hoc interfaces} or AHIs.

Being \emph{ad hoc} generelly means something made for a specific task or creating meaning in changing contexts.
The word itself originates from the Latin language and literally means \emph{for this} (situation). 
In modern english the two word combination is regarded as a single word and the Oxford Dictionary\footnote{http://oxforddictionaries.com/definition/english/ad-hoc} defines it as

\begin{quotation}
\textbf{Ad hoc}  /ad 'h\textturnscripta k/

created or done for a particular purpose as necessary
\end{quotation}

The word is used in many different fields and context.
For example in wireless networking, which is an important element of ubiquitous computing, ad hoc means not to be dependent on preexisting infrastructure.
This means that the network is highly dynamic and all participating nodes in the network graph are more or less doing routing for each other, which makes the network adaptive to changing contexts.

In graphic user interfaces, GUIs, \todo{GUI example} 



\subsubsection{Definition} 
We define ad hoc interfaces as \todo{physical} interfaces that can be created or accessed on demand for a particular purpose in mind. 

In this sense there is a degree of temporality embedded in the definition, in that the creation is somewhat impromptu and the purpose is non-continuing.

The questions is, how do we apply ad hoc elements or characteristics to physical interfaces.
The physical world comes with many constraints compared to the digital.
We can't just magically make physical objects appear and disappear which is to do on a digital display.
\todo{But what can we do} 

\subsubsection{Products with ad hoc characteristics}
Ad hoc characteristics in interfaces are not some novel invention and they an be found in several existing products.
A simple example would be The Clapper~(figure~\ref{ch:adhoc:theclapper})~\cite{theclapperWIKIPEDIA} from the mid eighties, an electrical switch reacting to sounds in a specific frequency, tuned to claps, to turn a switch on and off respectively.
Here the interface is pervasive in the nearby environment and one interacts with it when needed after which it ``disappears'' again.

\begin{figure}[hb]
	\centering
  		\includegraphics[width=1in]{figures/theclapper}
	\caption[The Clapper, a sound activated electronic switch.]
   {The Clapper, a sound activated electronic switch.}
   \label{ch:adhoc:theclapper}
\end{figure}

Another example is the conceptual and somewhat futuristic product ShapePhone based on particle jamming, which will also be addressed in \autoref{ch:jamming}(figure~\ref{fig:ch:jamming:jui-phone}).
ShapePhone is generic shape changing product which changes it behaviour based on its physical form.
For example, in its base form it is a phone but when wrapped around a wrist it could serve as a watch and when folded in some other way it could be a game controller.
So, the different interfaces are created on demand and have very different purposes of use.

In a masters course in 2011 called Innovation Project\footnote{https://services.brics.dk/java/courseadmin/INNOPRO} we, the authors, designed and implemented a dynamic shape shifting wall module with the ability to change its surface structure for acoustic regulation, i.e. diffusion, absorption and reflection for optimal conditions.
On top of that the module had illuminating features so that different areas could light up and serve as an aesthetic lighting for the atmosphere of the room. 
We envisioned two scenarios of use for the wall.
One where context awareness was the primary driver which meant that everything (acoustic and lighting conditions) would be automatically sensed and controlled.
The other where a user to a higher degree would be in control.
The user could actively create deformations on the surface for aesthetic purposes and also control illuminated areas by interacting with the wall surface, see \ref{fig:ch:adhoc:beomotion}. \todo{afrunding a paragraf} 

\begin{figure}
	\centering
	\begin{subfigure}{.46\textwidth}
		\centering
		\includegraphics[width=.9\linewidth]{figures/beomotion/prototype}
		\caption{Interactive prototype}
	\end{subfigure}%
	\begin{subfigure}{.54\textwidth}
		\centering
		\includegraphics[width=.9\linewidth]{figures/beomotion/concepts}
		\caption{Concept illustrations}
	\end{subfigure}
	\caption{BeoMotion product design, Innovation Project, Aarhus University 2011}
	\label{fig:ch:adhoc:beomotion}
\end{figure}

\subsubsection{Comparisons to other (X)UI} 
It might seem that AHI overlaps with the concept of Context Aware Computing in that they both have a focus on environmental context.
In context aware computing a system attempts to derive, through a variety of cues, what the current context of use is and as a result it adapts its behaviour \citep[chap. 8]{krumm2009ubiquitous}. 
In this way it is the system itself that takes action autonomously and the user continues on outside of the control loop. \todo{rephrase??}
Exactly the topic of control is one of the points where context aware systems have received criticism \cite{erickson2002some}, \citep[chap. 8]{krumm2009ubiquitous}.
The criticism has to do with a systems ability to make inference based on the analysis of quantitative contextual information available, something that can be quite difficult to do for a computer as contextual information is often subtle and implicit.
\todo{This is not supposed to be a critique of context aware systems as they surely have their legitimacy and offer a lot of convenience.}
\todo{An AHI can have CAC elements} 

Where context aware systems put the user out of the control loop AHIs do not.
In AHIs the user is the initiator of action.

It actually makes no sense to compare them directly as they cannot be juxtaposed.
They are both elements or characteristics of a system and one does not exclude the other as a part of a system.
It would seem quite natural to have an ad hoc system that also incorporates some context aware computing elements, for example \todo{give a good example} 
\blank
\todo{and TUIs}
\blank
\todo{and NUIs}  
\blank

\subsubsection{Discussion}

Some of the more interesting or ideal examples may seem 

\subsubsection{Noter \dots  } 
\begin{verbatim}


``fashioned from whatever is immediately available''
Ad hoc vs context aware
How does ad hoc interfaces differ from context aware interfaces (physical)

bringing together different aspects of various fields within HCI.

\end{verbatim} 
