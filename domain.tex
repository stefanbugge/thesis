Domestic setting:
 - Information appliances
 - Interactive household objects
 - Augmented furniture
 
Two broad categories for new interaction
 - New forms of Context Sensing
 - Embedded interactive technologies (part of the fabric of the space - we challenge this or do we)
 
Ubiquitous domestic environments are not build from scratch (7 challenges text)

Domestic environments are ever evolving - ever-changing nature of buildings, we look at Brand's "six S's"

We focus on Stuff and expand it

"Stuff seeks to be readily understood by the inhabitants"
"Absence of specialists and notations are notable in the case of stuff"


Ubiquitous computing in domestic environments

Efter bygninger er bygget, flytter folk ind. Først herefter tilpasser de bygningerne så de er til at leve i.

\subseciion{Ubiquitous domestic environments}
Buildings, like everything else, changes over time. But different parts of a building change with different rates, a wall will most likely endure longer that the paint on its sides and the ground on which the wall stands will most likely still be there when the wall collapses. This ever changing nature of buildings has been conceptualized by Brand in his book How Buildings Learn: What Happens After They’re Built (REF). Here Brand presents a framework, in which he define six S’s as a ground for understanding changes in a building - Site, Structure, Skin, Services, Space plan and Stuff. 
 
As shows in the figure, each layer describes a different rate of change, Site having the longest and Stuff the shortest rate of change.
Brand argues in favor for an approach where the inhabitants of a building can evolve and change the building over time according to their needs (BBC dokumentar). This is seen in contrast to a scenario where a single person or group designs a building for others to use. (Check dette i doku igen, for meget wiki)
Inspired by this concept of the building as a living, ever-evolving entity we want to explore and expand the level of Stuff – like the “six S’s”, can we create different layers of change within the realm of suff, having rates of change in days, hours or even minutes.
Before going into our own exploration of Stuff, we take a look at Rodden et al. (REF) as they have build upon Brands framework in the context of ubiquitous computing    
