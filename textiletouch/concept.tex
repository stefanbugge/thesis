%!TEX root = ../thesis.tex
The prototype that we have made for this chapter serves to exemplify both the qualities and possibilities of textile interfaces in general but also as an example of how surfaces in our environment can be enhanced and used in new ways for ad hoc interaction.
Our prototype is a generic touch surface meant to be integrated into the environment in various ways.
As the prototype has been made with textiles it is most suitable for textile based integrations such as furniture, sheets, cloth etc.
The technique used does however extend beyond the use of textiles and could just as well be implemented with other materials such as wallpaper, polymers, or printed circuit boards (PCB).

To explore more advanced touch interactions the prototype is programmed to recognise various gestures which we exemplify by a scenario of controlling existing devices of the home.
These gestures are then mapped to control specific functions of an implemented audio/video application. 

%The scenarios which we have envisioned are focus around pervasive touch surfaces in the home environment for the control of existing devices.