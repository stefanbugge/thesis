%!TEX root = ../thesis.tex
This section will go though the process and implementation of our prototype.
\todo{As one of the goals of this prototype has been to implement advanced interaction possibilities into a textile surface without the use of advanced machinery ... something something}
To give an insight into the prototype process as well as the final prototype, we have chosen to split the implementation into three overall iteration steps.
Our hope is that this will give a better understanding of the rationale behind our construction decisions as well as ... something something   

\subsection{Iteration 1}
The goal of our first iteration of the prototype was to ensure that the we had construction principles and the right materials for constructing a touch and pressure sensitive fabric.
The basic principle of a FSR sensor is to apply current to a semi-conductive material with piezoresistive capabilities and read the outgoing current from it.
As stress is applied to the semi-conductor the resistance will decrease and the outgoing current will therefore increase.
For our first prototype we applied this principle, inspired by (instructable), to construct a 2x3 pad in neoprene and a bend sensor.

\begin{verbatim}
- Fokus paa konstruktionspricipper
- Inspireret af instructable
- 2x3 pad i neopren
- Hvert felt virker som en individuel tryksensor
- Meget simpel at prototype, viste os at princippet med conductive 
  thread og plast virker
- Simpel grafisk gui til test
- Ikke skalerbar, man kan ikke trykke uden for felterne og hvert felt
  bruger et input i audrino (2x3=6)
- Opdateringshastighed svarende til max baud-rate
\end{verbatim}
\subsection{Iteration 2}
\begin{verbatim}
- Fokus paa skalerbarhed
- Inspireret af rSkin
- 7x7 pad i sofastof
- Row/column tilgang reducerer antallet af arduino input til 7+7(additiv)
  i stedet for 7x7(multiplikativ),
  men kraever mere kompleks arduino kode, er derfor lang mere skalerbar end foer
- Reset knap til at kalibrere efter deformation
- Har samme oploesning som stoerrelsen (7x7), tryk imellem linjerne vil derfor blive 
  registeret som enten den ene eller anden linje
- Opdateringshastighed svarende til max baud-rate
\end{verbatim}
\subsection{Iteration 3}
See \ref{app:textile-touch} for schematics and \dots
\begin{verbatim}
- Fokus paa kode (gestures,resolution,performance)
- Inspireret af UnMousePad
- Samme fysiske prototype som iteration 2
- Interpolering muliggoere tryk mellem linjerne, 10x saa hoej oploesning dog udfald 
  og varierende praesision (vis vi har eksperimenteret med forskellige oploesninger)
- Forskellige visualiseringer til performance evaluering
- Integrering af gesture recognition og test miljoe til dette
- Haptisk feedback, vibration
- Udfordringer: praesision, performance 
  (max baud-rate for hurtigt til at java kunne foelge med)
\end{verbatim}