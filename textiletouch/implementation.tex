%!TEX root = ../thesis.tex
hej
\todo{skal det praesenteres som iterationer eller som en proces?}
\subsection{Iteration 1}
\begin{verbatim}
- Fokus paa konstruktionspricipper
- Inspireret af instructable
- 2x3 pad i neopren
- Hvert felt virker som en individuel tryksensor
- Meget simpel at prototype, viste os at princippet med conductive 
  thread og plast virker
- Simpel grafisk gui til test
- Ikke skalerbar, man kan ikke trykke uden for felterne og hvert felt
  bruger et input i audrino (2x3=6)
- Opdateringshastighed svarende til max baud-rate
\end{verbatim}
\subsection{Iteration 2}
\begin{verbatim}
- Fokus paa skalerbarhed
- Inspireret af rSkin
- 7x7 pad i sofastof
- Row/column tilgang reducerer antallet af arduino input til 7+7(additiv)
  i stedet for 7x7(multiplikativ),
  men kraever mere kompleks arduino kode, er derfor lang mere skalerbar end foer
- Reset knap til at kalibrere efter deformation
- Har samme oploesning som stoerrelsen (7x7), tryk imellem linjerne vil derfor blive 
  registeret som enten den ene eller anden linje
- Opdateringshastighed svarende til max baud-rate
\end{verbatim}
\subsection{Iteration 3}
See \ref{app:textile-touch} for schematics and \dots
\begin{verbatim}
- Fokus paa kode (gestures,resolution,performance)
- Inspireret af UnMousePad
- Samme fysiske prototype som iteration 2
- Interpolering muliggoere tryk mellem linjerne, 10x saa hoej oploesning dog udfald 
  og varierende praesision (vis vi har eksperimenteret med forskellige oploesninger)
- Forskellige visualiseringer til performance evaluering
- Integrering af gesture recognition og test miljoe til dette
- Haptisk feedback, vibration
- Udfordringer: praesision, performance 
  (max baud-rate for hurtigt til at java kunne foelge med)
\end{verbatim}